\documentclass[12pt]{article}

\usepackage{graphics}
\usepackage{graphicx}
\usepackage{Sweave}
\usepackage{accents}

% New from euler:
\usepackage{ae}
\usepackage{color}
\usepackage{url}
\usepackage{hyperref}
\usepackage{setspace}

\topmargin=-0.5in
\textheight=9in
\textwidth=6.5in
\oddsidemargin=0in

%\usepackage{CJK}
%\usepackage{pinyin}
\def\E{\mathord{I\kern-.35em E}}
\def\R{\mathord{I\kern-.35em R}}
\def\P{\mathord{I\kern-.35em P}}
\def\I{\mathord{1\kern-.35em 1}}
\def\wt{\mathord{\widehat{\theta}}}

\newcommand{\proglang}[1]{\textbf{#1}}
\newcommand{\pkg}[1]{\texttt{\textsl{#1}}}
\newcommand{\code}[1]{\texttt{#1}}
\newcommand{\mg}[1]{{\textcolor {magenta} {#1}}}
\newcommand{\gr}[1]{{\textcolor {green} {#1}}}
\newcommand{\bl}[1]{{\textcolor {blue} {#1}}}

\newtheorem{thm}{Theorem}[section]
\newtheorem{myexplore}[thm]{Explore}
\newtheorem{mybackground}[thm]{Background}
\newtheorem{myquestion}[thm]{Question}
\newtheorem{myexample}[thm]{Example}
\newtheorem{mydefinition}[thm]{Definition}
\newtheorem{mytheorem}[thm]{Theorem}

\pagestyle{myheadings}    % Go for customized headings
\markboth{notused left title}{Michael Kane, Department of Statistics, Yale University \copyright 2010}
\newcommand{\sekshun}[1]                % In 'article' only the page
        {                               % number appears in the header.
        \section{#1}                    % I want the section name AND
        \markboth{#1 \hfill}{#1 \hfill} % the page, so I need a new kind
        }                               % of '\sekshun' command. 


\begin{document}

\doublespace

\setkeys{Gin}{width=1.0\textwidth} 



\begin{center}

%%%%%%%%%%%%%%%%%%%%%%%%%%%%%%%%%%%%%%%%%%%%
%%%%%%%%%%%%%%%%%%%%%%%%%%%%%%%%%%%%%%%%%%%%

{\Large\bf STAT 625 
%\vspace*{0.5cm}
%%%%%%%%%%%%%%%%%%%%%%%%%%%%%
Homework 6: Exploring the relationship between EPI and country 
characteristics\\
(Due Tuesday, November 30)\\

%%%%%%%%%%%%%%%%%%%%%%%%%%%%%

}

%%%%%%%%%%%%%%%%%%%%%%%%%%%%%%%%%%%%%%%%%%%%
%%%%%%%%%%%%%%%%%%%%%%%%%%%%%%%%%%%%%%%%%%%%


\end{center}

\begin{raggedright}
\parindent=0.5in

%%%%%%%%%%%%%%%%%%%%%%%%%%%%%%%%%%%%%%%%%%%%%%%%%%%%%%%%%%%%%%%%%%%%%%%%%%%%%%%%%%%%%%%%%%%%%%%%

%\section*{Scraping Jim Cramer's Mad Money Recap Page}

\noindent The Environmental Performance Index (EPI) ranks 163 countries on 25 
performance indicators tracked across ten policy categories covering both 
environmental public health and ecosystem vitality. These indicators provide a 
gauge at a national government scale of how close countries are to established 
environmental policy goals. For homework 6, explore and identify important 
associations between EPI and country characteristics.

The write-up should not assume that reader is 
familiar with the data.  The 
write-up should give background information describing the data; it should 
describe interesting aspects of the data found during your data exploration; 
interesting aspects should be visualized when appropriate; models used to 
show associations in the data should be described; and model residuals should 
be analyzed.

The data set, along with variable descriptions can be found on the 
Classes V2 server in Resources->Homework 6.  There are files listing variable 
names that may be of particular interest in the analysis.  All of the 
downloadable ending with .txt can be read into \proglang{R} using the
\code{dget} function.
%%%%%%%%%%%%%%%%%%%%%%%%%%%%%%%%%%%%%%%%%%%%%%%%%%%%%%%%%%%%%%%%%%%%%%%%%%%%%%%%%%%%%%%%%%%%%%%%

\end{raggedright}

\end{document}
