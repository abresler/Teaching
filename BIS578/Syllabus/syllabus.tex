\documentclass[12pt]{article}

\usepackage{graphics}
\usepackage{graphicx}
\usepackage{Sweave}
\usepackage{accents}

% New from euler:
\usepackage{ae}
\usepackage{color}
\usepackage{url}

\topmargin=-0.5in
\textheight=9in
\textwidth=6.5in
\oddsidemargin=0in

%\usepackage{CJK}
%\usepackage{pinyin}
\def\E{\mathord{I\kern-.35em E}}
\def\R{\mathord{I\kern-.35em R}}
\def\P{\mathord{I\kern-.35em P}}
\def\I{\mathord{1\kern-.35em 1}}
\def\wt{\mathord{\widehat{\theta}}}

\newcommand{\proglang}[1]{\textbf{#1}}
\newcommand{\pkg}[1]{\texttt{\textsl{#1}}}
\newcommand{\code}[1]{\texttt{#1}}
\newcommand{\mg}[1]{{\textcolor {magenta} {#1}}}
\newcommand{\gr}[1]{{\textcolor {green} {#1}}}
\newcommand{\bl}[1]{{\textcolor {blue} {#1}}}

\newtheorem{thm}{Theorem}[section]
\newtheorem{myexplore}[thm]{Explore}
\newtheorem{mybackground}[thm]{Background}
\newtheorem{myquestion}[thm]{Question}
\newtheorem{myexample}[thm]{Example}
\newtheorem{mydefinition}[thm]{Definition}
\newtheorem{mytheorem}[thm]{Theorem}

\pagestyle{myheadings}    % Go for customized headings
\markboth{notused left title}{Michael J. Kane, Yale University \copyright 2013}
\newcommand{\sekshun}[1]                % In 'article' only the page
        {                               % number appears in the header.
        \section{#1}                    % I want the section name AND
        \markboth{#1 \hfill}{#1 \hfill} % the page, so I need a new kind
        }                               % of '\sekshun' command. 


\begin{document}

\setkeys{Gin}{width=1.0\textwidth} 



\begin{center}

%%%%%%%%%%%%%%%%%%%%%%%%%%%%%%%%%%%%%%%%%%%%
%%%%%%%%%%%%%%%%%%%%%%%%%%%%%%%%%%%%%%%%%%%%

{\Large\bf BIS 578: Statistical Consulting\\
\vspace*{0.5cm}

%%%%%%%%%%%%%%%%%%%%%%%%%%%%%
Syllabus, Spring 2013\\
\vspace*{0.5cm}
Michael J. Kane\\
Yale University
%%%%%%%%%%%%%%%%%%%%%%%%%%%%%

}

%%%%%%%%%%%%%%%%%%%%%%%%%%%%%%%%%%%%%%%%%%%%
%%%%%%%%%%%%%%%%%%%%%%%%%%%%%%%%%%%%%%%%%%%%

\vspace*{1cm}


\end{center}

\begin{raggedright}
\parindent=0.5in

%%%%%%%%%%%%%%%%%%%%%%%%%%%%%%%%%%%%%%%%%%%%%%%%%%%%%%%%%%%%%%%%%%%%%%%%%%%%%%%%%%%%%%%%%%%%%%%%

\section{Course Description}

This class offers the chance for students to gain experience and practical
knowledge working as a biostatistician in a ``real-world'' setting.  Students
collaborate with an investigator, designing and implementing statistical tests
to further clinical research efforts, all under the supervision of the
instructor.  The class prepares students for further, unsupervised 
collaborations in their careers as biostatisticians.

\section{Learning Objectives}

\begin{enumerate}
\item Work with an investigator to understand the goals of a clinical study.
\item Contribute to a clinical study by designing and implementing statistical
  tests that answer questions posed by the study.
\item Gain experience in data cleaning, exploring, and analyzing a 
  ``real-world'' data set.
\item Summarize results and finding of statistical investigations and present
  these results in reports and presentations.
\end{enumerate}

\section{Prerequisite}

This class assumes that you are comfortable manipulating and visualization
data.  Furthermore, you be able to perform basic statistical tests,
run linear and generalized linear models, and understand the results.
You should have taken BIS 505 A/B or CDE 508 before taking this class, or
you will need to get the instructor's approval.

\section{Classes}

For the first two weeks of the semester students will meet twice a week for
lectures covering the development of research hypotheses, human subject 
protection, and determining sample size in statistical experiments -- vital
skills for any biostatistician working in a clinical setting.  Beginning 
in the third week students will begin collaborating with researchers in 
clinical investigations.  Students will meet with an instructor and
investigator weekly to give progress reports and receive guidance.
Additionally, students will sit in the research and design clinics
as well as the analytics clinics offered by the Yale Center for Analytical
Sciences.

\section{Grades}

Grades are determined by the student's ability to understand their assigned
investigation along with its goals; engage and contribute to the investigation;
and collaborate with the clinician to produce cutting-edge research results.
In the first week of class homework will be assigned to reinforce the 
materials presented in the lectures.  For rest of the semester students
will present their findings weekly to both the instructor and investigator.
Finally, students will provide written reports detailing their collaboration
with the clinical investigator.

\section{Schedule}

\begin{itemize}
\item Lecture 1: Developing research hypotheses and human subject protection
\item Lecture 2: Sample size (overview and theory)
\item Lecture 3: Sample size (user's guide)
\item Lecture 4: Investigations are presented and selected
\item Lectures 5-20: Present findings to investigator and instructor; and 
  attend research and design, and analytics clinics.
\item Final Presentation
\end{itemize}

\end{raggedright}

\end{document}
