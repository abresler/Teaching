% $Header: /cvsroot/latex-beamer/latex-beamer/solutions/conference-talks/conference-ornate-20min.en.tex,v 1.7 2007/01/28 20:48:23 tantau Exp $
\documentclass[13pt]{beamer}

\newcommand{\strong}[1]{{\normalfont\fontseries{b}\selectfont #1}}
\newcommand{\class}[1]{\mbox{\textsf{#1}}}
\newcommand{\func}[1]{\mbox{\texttt{#1()}}}
\newcommand{\code}[1]{\mbox{\texttt{#1}}}
\newcommand{\pkg}[1]{\strong{#1}}
\newcommand{\samp}[1]{`\mbox{\texttt{#1}}'}
\newcommand{\proglang}[1]{\textsf{#1}}
\newcommand{\putat}[3]{\begin{picture}(0,0)(0,0)\put(#1,#2){#3}\end{picture}}


\usepackage{beamerthemeAmsterdam}
\usepackage{setspace}
\usepackage{listings}
\usepackage{multirow}
\usepackage{relsize}
\usepackage{natbib}
\renewcommand{\bibsection}{\subsubsection*{\bibname } }

\lstset{basicstyle=\ttfamily}
%\lstset{language=R}
\lstset{tabsize=2}
\lstset{showstringspaces=false}

%\{onehalfspacing}
%\onehalfspacing
\usepackage{Sweave}

% This file is a solution template for:

% - Talk at a conference/colloquium.
% - Talk length is about 20min.
% - Style is ornate.



% Copyright 2004 by Till Tantau <tantau@users.sourceforge.net>.
%
% In principle, this file can be redistributed and/or modified under
% the terms of the GNU Public License, version 2.
%
% However, this file is supposed to be a template to be modified
% for your own needs. For this reason, if you use this file as a
% template and not specifically distribute it as part of a another
% package/program, I grant the extra permission to freely copy and
% modify this file as you see fit and even to delete this copyright
% notice. 


\mode<presentation>
{
%  \usetheme{Montpellier}
  \usetheme{Amsterdam}

%  \usetheme{Antibes}
  % or ...

  % \setbeamercovered{transparent}
  % or whatever (possibly just delete it)
}


\usepackage[english]{babel}
% or whatever

\usepackage[latin1]{inputenc}
% or whatever

\usepackage{times}
\usepackage[T1]{fontenc}
\usepackage[absolute,overlay]{textpos}
\newenvironment{reference}[2]{
  \begin{textblock*}{\textwidth}(#1,#2)
    \footnotesize\it\bgroup\color{black!50!black}}{\egroup\end{textblock*}}

\usepackage{graphics}
% Or whatever. Note that the encoding and the font should match. If T1
% does not look nice, try deleting the line with the fontenc.


%\title{Babelstream\smaller{\smaller{\smaller{{\texttrademark}}}}: 
\title{BIS 578 Lecture 1}



\author{Michael J. Kane}
% - Give the names in the same order as the appear in the paper.
% - Use the \inst{?} command only if the authors have different
%   affiliation.

\date{}

%\institute
%{
%  \inst{1}
%  Yale Center for Analytical Sciences, Yale University
%  \and 
%  \inst{2}
%  Department of Statistics, Yale University
%}

% - Use the \inst command only if there are several affiliations.
% - Keep it simple, no one is interested in your street address.

% - Either use conference name or its abbreviation.
% - Not really informative to the audience, more for people (including
%   yourself) who are reading the slides online

%\subject{Theoretical Computer Science}
% This is only inserted into the PDF information catalog. Can be left
% out. 



% If you have a file called "university-logo-filename.xxx", where xxx
% is a graphic format that can be processed by latex or pdflatex,
% resp., then you can add a logo as follows:

% \pgfdeclareimage[height=0.5cm]{university-logo}{university-logo-filename}
% \logo{\pgfuseimage{university-logo}}


% Delete this, if you do not want the table of contents to pop up at
% the beginning of each subsection:
%\AtBeginSubsection[]
%{
%  \begin{frame}<beamer>{Outline}
%    \tableofcontents[currentsection,currentsubsection]
%  \end{frame}
%}
%\AtBeginSection[]
%{
%  \begin{frame}<beamer>{Outline}
%    \tableofcontents[currentsection,currentsubsection]
%  \end{frame}
%}



% If you wish to uncover everything in a step-wise fashion, uncomment
% the following command: 

%\beamerdefaultoverlayspecification{<+->}


\begin{document}

\begin{frame}
  \titlepage
%  \begin{reference}{10mm}{78mm}
%    This is an expanded version of a talk given I gave at the 2010 
%    \proglang{R}Finance conference. 
%    The original can be downloaded from Bryan Lewis's website at 
%    \url{http://illposed.net/LewisKaneRInFinance.pdf}.
%  \end{reference}
\end{frame}

\begin{frame}{Outline}
  \tableofcontents
\end{frame}

% Structuring a talk is a difficult task and the following structure
% may not be suitable. Here are some rules that apply for this
% solution: 

% - Exactly two or three sections (other than the summary).
% - At *most* three subsections per section.
% - Talk about 30s to 2min per frame. So there should be between about
%   15 and 30 frames, all told.

% - A conference audience is likely to know very little of what you
%   are going to talk about. So *simplify*!
% - In a 20min talk, getting the main ideas across is hard
%   enough. Leave out details, even if it means being less precise than
%   you think necessary.
% - If you omit details that are vital to the proof/implementation,
%   just say so once. Everybody will be happy with that.

\section{Collaborating with an investigator}

\begin{frame}{Investigators need to collaborate with us}
``Clinical researchers rely on biostatisticians in order to design, conduct and 
analyse observational and experimental studies involving populations of 
subjects.'' \citep{Bangdiwala2001}
\end{frame}

\begin{frame}{We increase the rigor of investigations}
``Early collaboration between the clinical investigator and statistician can 
improve the study design and validity of the results by developing the 
statistical methodology that specifically addresses the research hypothesis.''
\cite{Adams2009}
\end{frame}

\begin{frame}{How dow we do this?}
\begin{itemize}
\items Graphs
\items Charts
\item Sample size and power analyses
\item Randomization and blinding procedures
\item Interim analysis
\item Data monitoring plans
\item Statistical analysis plan and implementation
\end{itemize} 
\end{frame}

\begin{frame}[allowframebreaks]{Bibliography}
\bibliographystyle{alpha}
\bibliography{references.bib}
\end{frame}

\end{document}


